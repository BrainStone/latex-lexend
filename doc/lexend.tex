\documentclass[oneside,a4paper]{l3doc}

% Packages
\usepackage{lexend}

\usepackage[english]{babel}
\usepackage{blindtext}
\usepackage{hologo}
\usepackage{listings}
\usepackage{pgffor}
\usepackage{soul}
\usepackage{xcolor}

% Settings
\setmonofont{DejaVu Sans Mono}

\hypersetup{
	hidelinks,
	bookmarksnumbered,
	bookmarksopen,
	bookmarksdepth=3,
	colorlinks=true,
	allcolors=blue,
	unicode
}

\urlstyle{rm}

\definecolor{lightgray}{gray}{0.95}

\lstset{breaklines=true,backgroundcolor=\color{lightgray},basicstyle=\ttfamily,frame=single,numbers=left,numberstyle=\ttfamily,aboveskip=\baselineskip}

% Additional commands
\DeclareRobustCommand{\code}[1]{{\sethlcolor{lightgray}\texttt{\hl{#1}}}}
\soulregister{\code}{1}
\makeatletter
\pdfstringdefDisableCommands{\let\code\@firstofone}
\makeatother

\providecommand{\tightlist}{%
  \setlength{\itemsep}{0pt}\setlength{\parskip}{0pt}}

% Document start
\begin{document}

\title{
	The \textsf{lexend} package\\
	\large{The \href{https://www.lexend.com/}{\textsf{Lexend} fonts} for \hologo{XeLaTeX} and
	\hologo{LuaLaTeX} through \textsf{fontspec}}
}
\author{
	Yannick Schinko\\
	\texorpdfstring{\url{https://github.com/BrainStone/latex-lexend}}{}
}
\date{
	\today\\
	v<VERSION>
}

\pagenumbering{roman}
\maketitle

\bigskip

\begin{abstract}
The purpose of this package is pretty straight forward:\\
The Lexend font collection has been designed by Dr. Bonnie Shaver-Troup and Thomas Jockin to make
reading easier for everyone.\\
Now my goal is it to bring this wonderful collection to world of \hologo{LaTeX}.
\end{abstract}

\bigskip
\tableofcontents
\newpage

\pagenumbering{arabic}

\section{Changelog}

<CHANGELOG>

\section{Acknowledgements}

I would like to take the time to thank everyone that helped me along my journey of creating this
package. It would not have been possible without every single one of you.

Special thanks to David Carlisle, Ulrike Fischer and Phelype Oleinik who helped me tremendously and
continuously in the \hologo{LaTeX} Stack Exchange chat. I really can't put it in words how much your
continued contributions and willingness to help helped me during the creation of this package.

Special thanks to Dr. Bonnie Shaver-Troup and Thomas Jockin for creating these fonts and allowing
people to have an easier time reading.

Consider this contribution to the great \hologo{LaTeX} community as my thank you to you all.

\section{Package Loading and Options}

\subsection{Loading behavior}

First of all this package depends on the \textsf{fontspec} package and therefore requires
\hologo{XeLaTeX} and \hologo{LuaLaTeX}. So when you load this package the \textsf{fontspec} will
also be loaded.

When this package is loaded it sets the main font to \textsf{LexendDeca} and the sans font to
\textsf{LexendGiga} using \textsf{fontspec}'s \code{\textbackslash{}setmainfont} and
\code{\textbackslash{}setsansfont} commands.

If you do wish to change the default fonts, just override the set fonts yourself with the same
commands.

\subsection{Package Options}

\textit{None at the moment}

\section{Features}

\subsection{Commands}

\subsection{.fontspec Files}

Every Lexend font has its own \code{.fontspec} file. Though they are all identical (excluding the
actual name) and look like this (\code{<FONTNAME>} will be replaced with the font's name of course):

\lstinputlisting[language={[LaTeX]TeX},caption={template.fontspec}]{template.fontspec}

\subsubsection{Expanding the Default Fontspecs}

If you want to keep the default font settings but would like to add onto them, that's really easy:\\
Use the \code{\textbackslash{}defaultfontfeatures+} command of \textsf{fontspec}.

\bigskip

For example if you would like to set the default color for every time you use the font
\textsf{LexendPeta}, all you have to do is this:

\begin{lstlisting}[language={[LaTeX]TeX}]
\defaultfontfeatures+[LexendPeta]{
	Color=888888
}
\end{lstlisting}

Since this is handled by the \textsf{fontspec} package, please refer to their documentation at
\url{https://ctan.org/pkg/fontspec}.

\subsection{Future Features}

This package is still very young and also fairly incomplete. There's a list of things I'd like to
add over time. Ordered by how soon I wish to implement them myself.

\begin{itemize}
\item Package option to turn off setting the main and sans font\tightlist
\item Support for the continuous spacing font variant of Lexend\tightlist
\item Support for other \hologo{LaTeX} engines\\
	That's something I definately need help with. So feel free to reach out to me or to create a
	Pull Request on GitHub.\tightlist
\end{itemize}

\section{License}

The licensing of this project is a bit uncommon, so let me explain:

The final package, the one you find on \href{https://ctan.org/}{CTAN}, is licensed under
\href{https://ctan.org/license/lppl1.3c}{The \hologo{LaTeX} Project Public License 1.3c}. So use it
like you would use most other packages on \href{https://ctan.org/}{CTAN}.

The source repository, the one you find under \url{https://github.com/BrainStone/latex-lexend},
however is licensed under the \href{https://www.gnu.org/licenses/gpl-3.0.html}{GNU General Public
License v3.0}.

Lastly, the font collection itself is licensed under the
\href{https://opensource.org/licenses/OFL-1.1}{SIL Open Font License 1.1}. You can find the license
file under \url{https://github.com/ThomasJockin/lexend/blob/master/OFL.txt}.

\section{Font Samples}

While this whole document uses \textsf{LexendDeca} as the main font and \textsf{LexendGiga} as the
sans font, it doesn't hurt showing them off individually.\\
Each font will be shown off in the sizes \small\textbackslash{}small\normalsize{},
\textbackslash{}normalsize and \large\textbackslash{}large\normalsize{} and the variants
\textbf{\textbackslash{}textbf}, \textit{\textbackslash{}textit} and
\textbf{\textit{\textbackslash{}textbf\{\textbackslash{}textit\}}}

\foreach \fontVariant in \lexendVariants
{
	\subsection{\texorpdfstring{\fontspec{\fontVariant}{\fontVariant}}{\fontVariant}}
	
	\fontspec{\fontVariant}{
		\small\blindtext\normalsize\bigskip

		\blindtext\bigskip

		\large\blindtext\normalsize\bigskip

		\textbf{\blindtext}\bigskip

		\textit{\blindtext}\bigskip

		\textbf{\textit{\blindtext}}
	}
}

\end{document}
